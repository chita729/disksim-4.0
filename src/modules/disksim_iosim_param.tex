\noindent 
\begin{tabular}{|p{\lpmodwidth}|p{\lpnamewidth}|p{0.5in}|p{0.5in}|}
\cline{1-4}
\texttt{disksim\_iosim} & \texttt{I/O Trace Time Scale} & float & optional \\ 
\cline{1-4}
\multicolumn{4}{|p{6in}|}{
This specifies a value by which each arrival time in a trace is
multiplied. For example, a value of 2.0 doubles each arrival time,
lightening the workload by stretching it out over twice the length of
time. Conversely, a value of 0.5 makes the workload twice as heavy by
compressing inter-arrival times. This value has no effect on
workloads generated internally (by the synthetic generator).
}\\ 
\cline{1-4}
\multicolumn{4}{p{5in}}{}\\
\end{tabular}\\ 
\noindent 
\begin{tabular}{|p{\lpmodwidth}|p{\lpnamewidth}|p{0.5in}|p{0.5in}|}
\cline{1-4}
\texttt{disksim\_iosim} & \texttt{I/O Mappings} & list & optional \\ 
\cline{1-4}
\multicolumn{4}{|p{6in}|}{
This is a list of \texttt{iomap} blocks (see below) which enable translation
of disk request sizes and locations in an input trace into disk
request sizes and locations appropriate for the simulated environment.
When the simulated environment closely matches the traced environment,
these mappings may be used simply to reassign disk device numbers.
However, if the configured environment differs significantly from the
trace environment, or if the traced workload needs to be scaled (by
request size or range of locations), these mappings can be used to
alter the the traced ``logical space'' and/or scale request sizes and
locations. One mapping is allowed per traced device.
The mappings from devices identified in the trace to the storage
subsystem devices being modeled are provided by block values.
}\\ 
\cline{1-4}
\multicolumn{4}{p{5in}}{}\\
\end{tabular}\\ 
