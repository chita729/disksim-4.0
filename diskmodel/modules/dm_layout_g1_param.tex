\noindent 
\begin{tabular}{|p{\lpmodwidth}|p{\lpnamewidth}|p{0.5in}|p{0.5in}|}
\cline{1-4}
\texttt{dm\_layout\_g1} & \texttt{LBN-to-PBN mapping scheme} & int & required \\ 
\cline{1-4}
\multicolumn{4}{|p{6in}|}{
This specifies the type of LBN-to-PBN mapping used by the disk.
0 indicates that the conventional mapping scheme is used:
LBNs advance along the 0th track of the 0th cylinder, then along the
1st track of the 0th cylinder, thru the end of the 0th cylinder, then
to the 0th track of the 1st cylinder, and so forth.
1 indicates that the conventional mapping scheme is modified slightly,
such that cylinder switches do not involve head switches. Thus, after
LBNs are assigned to the last track of the 0th cylinder, they are
assigned to the last track of the 1st cylinder, the next-to-last track
of the 1st cylinder, thru the 0th track of the 1st cylinder. LBNs are
then assigned to the 0th track of the 2nd cylinder, and so on
(``first cylinder is normal'').
2 is like 1 except that the serpentine pattern does not reset at the
beginning of each zone; rather, even cylinders are always ascending and
odd cylinders are always descending.
}\\ 
\cline{1-4}
\multicolumn{4}{p{5in}}{}\\
\end{tabular}\\ 
\noindent 
\begin{tabular}{|p{\lpmodwidth}|p{\lpnamewidth}|p{0.5in}|p{0.5in}|}
\cline{1-4}
\texttt{dm\_layout\_g1} & \texttt{Sparing scheme used} & int & required \\ 
\cline{1-4}
\multicolumn{4}{|p{6in}|}{
This specifies the type of sparing used by the disk. Later parameters determine
where spare space is allocated.
0~indicates that no spare sectors are allocated.
1~indicates that entire tracks of spare sectors are allocated at the ``end''
of some or all zones (sets of cylinders).
2~indicates that spare sectors are allocated at the ``end'' of each cylinder.
3~indicates that spare sectors are allocated at the ``end'' of each track.
4~indicates that spare sectors are allocated at the ``end'' of each cylinder
and that slipped sectors do not utilize these spares (more spares are located
at the ``end'' of the disk).
5~indicates that spare sectors are allocated at the ``front'' of each cylinder.
6~indicates that spare sectors are allocated at the ``front'' of each cylinder
and that slipped sectors do not utilize these spares (more spares are located
at the ``end'' of the disk).
7~indicates that spare sectors are allocated at the ``end'' of the disk.
8~indicates that spare sectors are allocated at the ``end'' of each range
of cylinders.
9~indicates that spare sectors are allocated at the ``end'' of each zone.
10~indicates that spare sectors are allocated at the ``end'' of each zone
and that slipped sectors do not use these spares (more spares are located
at the ``end'' of the disk).
}\\ 
\cline{1-4}
\multicolumn{4}{p{5in}}{}\\
\end{tabular}\\ 
\noindent 
\begin{tabular}{|p{\lpmodwidth}|p{\lpnamewidth}|p{0.5in}|p{0.5in}|}
\cline{1-4}
\texttt{dm\_layout\_g1} & \texttt{Rangesize for sparing} & int & required \\ 
\cline{1-4}
\multicolumn{4}{|p{6in}|}{
This specifies the range (e.g., of cylinders) over which spares are
allocated and maintained. Currently, this value is relevant only for
disks that use ``sectors per cylinder range'' sparing schemes.
}\\ 
\cline{1-4}
\multicolumn{4}{p{5in}}{}\\
\end{tabular}\\ 
\noindent 
\begin{tabular}{|p{\lpmodwidth}|p{\lpnamewidth}|p{0.5in}|p{0.5in}|}
\cline{1-4}
\texttt{dm\_layout\_g1} & \texttt{Skew units} & string & optional \\ 
\cline{1-4}
\multicolumn{4}{|p{6in}|}{
This sets the units with which units are input: \texttt{revolutions} or
\texttt{sectors}. The ``disk-wide'' value set here may be overridden
per-zone. The default unit is \texttt{sectors}.
}\\ 
\cline{1-4}
\multicolumn{4}{p{5in}}{}\\
\end{tabular}\\ 
\noindent 
\begin{tabular}{|p{\lpmodwidth}|p{\lpnamewidth}|p{0.5in}|p{0.5in}|}
\cline{1-4}
\texttt{dm\_layout\_g1} & \texttt{Zones} & list & required \\ 
\cline{1-4}
\multicolumn{4}{|p{6in}|}{
This is a list of zone block values describing the zones/bands of the disk.
}\\ 
\cline{1-4}
\multicolumn{4}{p{5in}}{}\\
\end{tabular}\\ 
